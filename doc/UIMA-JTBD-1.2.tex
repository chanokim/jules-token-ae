\documentclass[11pt,a4paper,halfparskip]{scrartcl}
\usepackage{graphicx}
\usepackage[utf8x]{inputenc}
\usepackage{url} 
\usepackage[T1]{fontenc}
\usepackage{ucs} 
\pagestyle{plain}

\title{\small{Documentation for}\\\huge UIMA Wrapper for JULIE Token Boundary Detector
  (UIMA-JTBD-1.2)}

\author{\normalsize Katrin Tomanek\\
  \normalsize  Jena University Language \& Information Engineering (JULIE) Lab\\
  \normalsize F\"urstengraben 30 \\
  \normalsize D-07743 Jena, Germany\\
  {\normalsize \tt tomanek@coling-uni-jena.de} }


\date{}

\begin{document}
\maketitle

\section{Objective}

UIMA-JTBD is a sentence boundary detector for UIMA.  It is part of
the JULIE NLP tool suite\footnote{\url{http://www.julielab.de/}} which
contains several NLP components (all UIMA compliant) from sentence
splitting to named entity recognition and normalization as well as a
comprehensive UIMA type system.

UIMA-JTBD is currently available in version 1.2. UIMA-JTBD is an UIMA
wrapper for JTBD, the respective command-line version. For more
detailed information on the functioning of JTBD check the JTBD
documentation or refer to \cite{Tomanek2007a}.

\section{Installation}

UIMA-JTBD comes as a UIMA pear file. Run the Pear-Installer (e.g.,
\url{./runPearInstaller.sh} for Linux) from your UIMA-bin directory.
After installation, you will find a subfolder \url{desc} in you
installation folder. This directory contains a descriptor
\url{TokenAnnotator.xml} for UIMA-JTBD. You may now e.g. run UIMA's
Collection Proeccessing Engine Configurator (\url{cpeGUI.sh}) and add
UIMA-JTBD as a component into your NLP pipeline.

This pear package also contains a model for sentence splitting. The
model was trained on a special bio-medical corpus which consists of data from (manually
annotated) material which we took from MedLine abstracts and a
modified version of
\textsc{PennBioIE}'s\footnote{\url{http://bioie.ldc.upenn.edu/}}
underlying tokenization. In the \textsc{PennBioIE} corpus, some purely
alphanumeric strings are divided into smaller tokens to support
\textsc{PennBioIE}'s entity annotation, especially in a common
annotation for variation events (e.g.  "S45F" with "S"=state\_original,
"45"=location, "F"=state\_altered).  Those splits were (manually)
undone to fit our tokenization guidelines.
Currently, our tokenization corpus comprises about 36000 sentences.
An accuracy of ACC=96.7\% is reached on this data using 10-fold
cross-validation.  You will find the model trained on this data in the
directory \url{resources}.

%\section{Changelog}
% uncomment when needed


\section{Requirements and Dependencies}

% mostly our tools will be based on java 1.5 and use UIMA
UIMA-JTBD is written in Java 1.5 using Apache UIMA version
2.1.0-incubation\footnote{\url{http://incubator.apache.org/uima/}}. It
was not tested with other UIMA versions.

% ref to our type system
The input and output of an AE takes place by annotation objects. The
classes corresponding to these objects are part of a \emph{JULIE UIMA
  Type System}\footnote{The \emph{JULIE UIMA type system} can be
  obtained from \url{http://www.julielab.de/}}.

UIMA-JTBD-1.2 is based on JTBD-1.6 which employs the machine learning
toolkit MALLET \cite{mallet}.




\section{Using the AE -- Descriptor Configuration}
% carefully edit this section!

In UIMA, each component is configured by a descriptor in XML. In the
following we describe how the descriptor required by this AE can be
created with \emph{Component Descriptor Editor}, an Eclipse plugin
which is part of the UIMA SDK.

A descriptor contains information on different aspects. The following
subsection refers to each sub aspect of the descriptor which is, in
the Component Descriptor Editor, a separate \emph{tabbed page}. For an
indepth description of the respective configuration aspects or tabs,
please refer to the \emph{UIMA SKD User's
  Guide}\footnote{\url{http://incubator.apache.org/uima/}}, especially
chapter 12 on ``Component Descriptor Editor User's Guide''.

To define your own descriptor go through each tabbed pages mentioned
here, make your respective entries (especially in page \emph{Parameter
  Settings} you will be able to configure JNET to your needs) and save
the descriptor as \url{SomeName.xml}.

Otherwise, you can of course employ the descriptor that is contained
in the pear package you downloaded (in your installation directory, see
\url{desc/TokenAnnotator.xml}).

\paragraph{Overview}
This tab provides general informtion about the component. For the
UIMA-JTBD you need to provide the information as specified in Table
\ref{tab:overview}.
% adapt to your needs, remember to change values in tabular below!

\begin{table}[h!]
  \centering
  \begin{tabular}{|p{3.5cm}|p{4cm}|p{6cm}|}
    \hline
    Subsection & Key & Value \\
    \hline\hline
    Implementation Details & Implementation Language & Java \\
    \cline{2-3}
    & Engine Type & primitive \\
    \hline
    Runtime Information & updates the CAS & check \\
    \cline{2-3}
    & multiple deployment allowed & check \\
    \cline{2-3}
    & outputs new CASes &  don't check \\
    \cline{2-3}
    & Name of the Java class file & \url{de.julielab.jules.ae.TokenAnnotator}\\
    \hline
    Overall Identification Information & Name &  Token Annotator \\
    \cline{2-3}
    & Version &  1.2 \\
    \cline{2-3}
    & Vendor & JULIE Lab\\
    \cline{2-3}
    & Description & not needed\\
    \hline
  \end{tabular}
  \caption{Overview/General Settings for AE.}
  \label{tab:overview}
\end{table}


\paragraph{Aggregate}
% for primitive AEs this does not have to be set
Not needed here, as this AE is a primitive.

\paragraph{Parameters}
\label{sss:parameters}
% adapt to your needs

See Table \ref{tab:parameters} for a specification of the
configuration parameters of this AE. Do not check ``Use Parameter
Groups'' in this tab.

\begin{table}[h!]
  \centering
  \begin{tabular}{|p{4cm}|p{2cm}|p{2cm}|p{2cm}|p{4cm}|}
    \hline
    Parameter Name & Parameter Type & Mandatory & Multivalued & Description \\
    \hline\hline
    ModelFilename & String & yes & no & filename of trained model for
    JTBD\\
    \hline
  \end{tabular}
  \caption{Parameters of this AE.}
  \label{tab:parameters}
\end{table}


\paragraph{Parameter Settings}
\label{sss:param_settings}
% adapt to your needs

The specific parameter settings are filled in here. For each of the
parameters defined in \ref{sss:parameters}, add the respective values
here (has to be done at least for each parameter that is defined as
mandatory). See Table \ref{tab:param_settings} for the respective
parameter settings of this AE.

\begin{table}[h!]
  \centering
  \begin{tabular}{|p{4cm}|p{4cm}|p{7cm}|}
    \hline
    Parameter Name & Parameter Syntax & Example \\
    \hline\hline
    ModelFilename & full path & \url{/home/tomanek/JULIE_token.mod}\\
    \hline
  \end{tabular}
  \caption{Parameter settings of this AE.}
  \label{tab:param_settings}
\end{table}

\paragraph{Type System}
\label{sss:type_system}
On this page, go to \emph{Imported Type} and add the \emph{JULIE UIMA
  Type System}. (Use ``Import by Location'').


\paragraph{Capabilities}
\label{sss:capabilities}
The tokenizer takes as input annotations from type \url{de.julielab.jules.types.Sentence} and returns annotations from type \url{de.julielab.jules.types.Token}. See Table \ref{tab:capabilities}.
% adapt if needed
\begin{table}[h!]
  \centering
  \begin{tabular}{|p{5cm}|p{2cm}|p{2cm}|}
    \hline
    Type & Input & Output \\
    \hline\hline
     de.julielab.jules.types.Sentence & $\surd$ & \\
      \hline
     de.julielab.jules.types.Token & &  $\surd$  \\
      \hline
  \end{tabular}
  \caption{Capabilities of this AE.}
  \label{tab:capabilities}
\end{table} 


\paragraph{Index}
% adapt if needed
Nothing needs to be done here.

\paragraph{Resources}
% adapt if needed
Nothing needs to be done here.


\section{Copyright and License}
% leave unchanged
This software is Copyright (C) 2007 Jena University Language \& Information
Engineering Lab (Friedrich-Schiller University Jena, Germany), and is
licensed under the terms of the Common Public License, Version 1.0 or (at
your option) any subsequent version.

The license is approved by the Open Source Initiative, and is
available from their website at \url{http://www.opensource.org}.

\bibliographystyle{alpha}
\bibliography{/home/tomanek/coling/biblio/literature.bib}


\end{document}
