\documentclass[11pt,a4paper,halfparskip]{scrartcl}
\usepackage{graphicx}
\usepackage[utf8x]{inputenc}
\usepackage{url} 
\usepackage[T1]{fontenc}
\usepackage{ucs} 
\pagestyle{plain}

\title{\small{Documentation for}\\\huge JULIE Lab Token Boundary Detector\\\vspace{3mm}\small{Version 2.4}}


\author{\normalsize Katrin Tomanek\\
  \normalsize  Jena University Language \& Information Engineering (JULIE) Lab\\
  \normalsize F\"urstengraben 30 \\
  \normalsize D-07743 Jena, Germany\\
  {\normalsize \tt katrin.tomanek@uni-jena.de} }




\date{}

\begin{document}
\maketitle

\section{UIMA Wrapper}

The JULIE Lab Token Boundary Detector (UIMA-JTBD) is a token boundary
detector for UIMA.  It is part of the JULIE Lab NLP tool
suite\footnote{\url{http://www.julielab.de/}} which contains several
NLP components (all UIMA compliant) from sentence splitting to named
entity recognition and normalization as well as a comprehensive UIMA
type system.

UIMA-JTBD is an UIMA wrapper for JTBD, the respective command-line
version. For more detailed information on the functioning of JTBD
check the JTBD documentation or refer to \cite{Tomanek2007a}.

\subsection{Installation}

UIMA-JTBD comes as a UIMA pear file. Run the Pear-Installer (e.g.,
\url{./runPearInstaller.sh} for Linux) from your UIMA-bin directory.
After installation, you will find a subfolder \url{desc} in you
installation folder. This directory contains a descriptor
\url{TokenAnnotator.xml} for UIMA-JTBD. You may now e.g. run UIMA's
Collection Proeccessing Engine Configurator (\url{cpeGUI.sh}) and add
UIMA-JTBD as a component into your NLP pipeline.

This pear package also contains a model for tokenization splitting. The
model was trained on a special bio-medical corpus which consists of
data from (manually annotated) material which we took from MedLine
abstracts and a modified version of
\textsc{PennBioIE}'s\footnote{\url{http://bioie.ldc.upenn.edu/}}
underlying tokenization. In the \textsc{PennBioIE} corpus, some purely
alphanumeric strings are divided into smaller tokens to support
\textsc{PennBioIE}'s entity annotation, especially in a common
annotation for variation events (e.g.  "S45F" with
"S"=state\_original, "45"=location, "F"=state\_altered).  Those splits
were (manually) undone to fit our tokenization guidelines.  Currently,
our tokenization corpus comprises about 36000 sentences.  An accuracy
of ACC=96.7\% is reached on this data using 10-fold cross-validation.
You will find the model trained on this data in the directory
\url{resources}.

%\section{Changelog}
% uncomment when needed


\subsection{Requirements and Dependencies}

% mostly our tools will be based on java 1.5 and use UIMA
UIMA-JTBD is written in Java (version 1.5 or above required) using
Apache UIMA version
2.2.x-incubation\footnote{\url{http://incubator.apache.org/uima/}}.

% ref to our type system
The input and output of an AE takes place by annotation objects. The
classes corresponding to these objects are part of the \emph{JULIE Lab
  UIMA Type System} in its current version (2.1).\footnote{The
  \emph{JULIE Lab UIMA type system} can be separately obtained from
  \url{http://www.julielab.de/}, however, this package already
  includes the necessary parts of the type system.}

UIMA-JTBD employs the machine learning toolkit MALLET
\cite{McCallum2002}.




\subsection{Using the AE -- Descriptor Configuration}
% carefully edit this section!

In UIMA, each component is configured by a descriptor in XML. In the
following we describe how the descriptor required by this AE can be
created with the \emph{Component Descriptor Editor}, an Eclipse plugin
which is part of the UIMA SDK.

A descriptor contains information on different aspects. The following
subsection refers to each sub aspect of the descriptor which is, in
the Component Descriptor Editor, a separate \emph{tabbed page}. For an
indepth description of the respective configuration aspects or tabs,
please refer to the \emph{UIMA SKD User's
  Guide}\footnote{\url{http://incubator.apache.org/uima/}}, especially
the chapter on ``Component Descriptor Editor User's Guide''.

To define your own descriptor go through each tabbed pages mentioned
here, make your respective entries (especially in page \emph{Parameter
  Settings} you will be able to configure JNET to your needs) and save
the descriptor as \url{SomeName.xml}.

Otherwise, you can of course employ the descriptor that is contained
in the pear package you downloaded (in your installation directory, see
\url{desc/TokenAnnotator.xml}).

\paragraph{Overview}
This tab provides general informtion about the component. For the
UIMA-JTBD you need to provide the information as specified in Table
\ref{tab:overview}.
% adapt to your needs, remember to change values in tabular below!

\begin{table}[h!]
  \centering
  \begin{tabular}{|p{3.5cm}|p{4cm}|p{6cm}|}
    \hline
    Subsection & Key & Value \\
    \hline\hline
    Implementation Details & Implementation Language & Java \\
    \cline{2-3}
    & Engine Type & primitive \\
    \hline
    Runtime Information & updates the CAS & yes \\
    \cline{2-3}
    & multiple deployment allowed & yes \\
    \cline{2-3}
    & outputs new CASes &  no \\
    \cline{2-3}
    & Name of the Java class file & \url{de.julielab.jules.ae.TokenAnnotator}\\
    \hline
    Overall Identification Information & Name &  Token Annotator \\
    \cline{2-3}
    & Version &  2.4 \\
    \cline{2-3}
    & Vendor & JULIE Lab\\
    \cline{2-3}
    & Description & not needed\\
    \hline
  \end{tabular}
  \caption{Overview/General Settings for AE.}
  \label{tab:overview}
\end{table}


\paragraph{Aggregate}
% for primitive AEs this does not have to be set
Not needed here, as this AE is a primitive.

\paragraph{Parameters}
\label{sss:parameters}
% adapt to your needs

See Table \ref{tab:parameters} for a specification of the
configuration parameters of this AE. Do not check ``Use Parameter
Groups'' in this tab.

\begin{table}[h!]
  \centering
  \begin{tabular}{|p{4cm}|p{2cm}|p{2cm}|p{2cm}|p{4cm}|}
    \hline
    Parameter Name & Parameter Type & Mandatory & Multivalued & Description \\
    \hline\hline
    ModelFilename & String & yes & no & filename to model trained for JTBD\\
    \hline
  \end{tabular}
  \caption{Parameters of this AE.}
  \label{tab:parameters}
\end{table}


\paragraph{Parameter Settings}
\label{sss:param_settings}
% adapt to your needs

The specific parameter settings are filled in here. For each of the
parameters defined in \ref{sss:parameters}, add the respective values
here (has to be done at least for each parameter that is defined as
mandatory). See Table \ref{tab:param_settings} for the respective
parameter settings of this AE.

\begin{table}[h!]
  \centering
  \begin{tabular}{|p{4cm}|p{4cm}|p{7cm}|}
    \hline
    Parameter Name & Parameter Syntax & Example \\
    \hline\hline
    ModelFilename & full path & \url{resources/JULIE_life-science-1.6.mod.gz}\\
    \hline
  \end{tabular}
  \caption{Parameter settings of this AE.}
  \label{tab:param_settings}
\end{table}

\paragraph{Type System}
\label{sss:type_system}
On this page, go to \emph{Imported Type} and add the \emph{JULIE UIMA
  Type System}. (Use ``Import by Location'').


\paragraph{Capabilities}
\label{sss:capabilities}
The tokenizer takes as input annotations from type \url{de.julielab.jules.types.Sentence} and returns annotations from type \url{de.julielab.jules.types.Token}. See Table \ref{tab:capabilities}.
% adapt if needed
\begin{table}[h!]
  \centering
  \begin{tabular}{|p{5cm}|p{2cm}|p{2cm}|}
    \hline
    Type & Input & Output \\
    \hline\hline
     de.julielab.jules.types.Sentence & $\surd$ & \\
      \hline
     de.julielab.jules.types.Token & &  $\surd$  \\
      \hline
  \end{tabular}
  \caption{Capabilities of this AE.}
  \label{tab:capabilities}
\end{table} 


\paragraph{Index}
% adapt if needed
Nothing needs to be done here.

\paragraph{Resources}
% adapt if needed
Nothing needs to be done here.



\clearpage
\section{JTBD: Core Functionality and Stand Alone Tool}
\label{sec_objective}


JULIE Token Boundary Detector (JTBD) is a tokenizer developed and
optimized for the bio-medical domain. In contrast to many other
tokenizers which consists of simple patterns, JTBD is based on machine
learning (see Section \ref{sec:background}) which enables it to handle
also tricky cases occuring frequently in life science documents. See
\cite{tomanek.medinfo07} for a more in depth description of JTDB and a
performance study.


JTBD offers the following functionalities:
\begin{itemize}
\item training a model
\item prediction using a previously trained model
\item evaluation
\end{itemize}

% \subsection{Changelog}
% since version 1.5:
% \begin{itemize}
% \item minor bugs removed (consecutive whitespaces)
% \end{itemize}

% since version 1.4:
% \begin{itemize}
% \item models compressed with GZIP
%   umlauts)\end{itemize}


% since version 1.4:
% \begin{itemize}
% \item regular expression set modified (JTBD can now also handle german
%   umlauts)\end{itemize}


\subsection{Installation}
Just unpack the tar-ball. The program is written in Java\footnote{Java
  is a registered trademark of Sun Microsystems, Inc.}. Note that JTBD
was only tested with Java 1.5; you need at least the Java 1.5 runtime
environment installed on your system to run JTBD. In addition to the
common Java libraries, JTBD employs \textsc{MALLET} \cite{mallet}, a
machine learning toolkit (no further installation steps are
required here).

\subsection{File Formats}
\label{sec_formats}


There are two input formats for text documents. There are some example
documents in directory \url{testdata}. All data need to be simple
ASCII.

\begin{itemize}
\item For training and evaluation, two files are needed, called
  <sent-file> and <tok-file> later on. <sent-file> contains sentence
  information (one sentence per line) and <tok-file> contains the
  tokenization information (also here one sentence per line and
  tokenization made by adding white spaces). See the files in
  \url{testdata} for examples (\url{train.*}).
\item For tokenization, the input files have to be sentence splitted,
  i.e. there has to be exactly one sentence per line.  See the files
  in \url{testdata} for examples (\url{split.txt}).
\end{itemize}

\subsection{Using JTBD}

To execute JTBD just type 

\begin{verbatim}
./runJTBDpackaged.sh
\end{verbatim}

without any arguments. This will print the following list of available
modes:

\begin{verbatim}
usage: JTBD <mode> <mode-specific-parameters>

Available modes:
c: check data 
s: 90-10 split evaluation
x: cross validation 
t: train a tokenizer 
p: predict with tokenizer 
e: evaluation on previously trained model
\end{verbatim}

When running JTBD only with the mode as its only parameter it will
return the specific parameters needed for this mode.

\subsection{Data Check}

The provided data is checked for the correct format.

\begin{verbatim}
./runJTBDpackaged.sh c

-> usage: JTBD c <sent-file> <tok-file>
\end{verbatim}

\textit{<sent-file>} is a sentence splitted document (one sentence in
each line); \textit{<tok-file>} is a sentence splitted and tokenized
document (one sentence in each line and a white space after each
single token).

\subsection{Evaluation}

There are two evaluation modes to evaluate the tokenizer on given
training material. Performance is measured in terms of accuracy, i.e.\
the number of correct decisions divided by the total number of
decisions being made.

\begin{description}
\item [90-10 split evaluation] the given data is split into two data
  sets, 90\% of the files are used for training the model, the other
  10\% are used to evaluate the trained model (splits are made on file
  level, i.e. you should make sure, that the files are more or less of
  the same size)

\begin{verbatim}
./runJTBDpackaged.sh s

-> usage: usage: JTBD s <sent-file> <tok-file> <predout-file> <errout-file>
\end{verbatim}

  \textit{<sent-file>} is a sentence splitted document (one sentence
  in each line); \textit{<tok-file>} is a sentence splitted and
  tokenized document (one sentence in each line and a white space
  after each single token); \textit{predout-file} is a file where all
  the predictions made during this evaluation are written to.
  \textit{<errorFile>} is a file where all predictions errors are
  written to. The Format of this file is as follows: @P->N indicates a
  false negative error, @N->P indicates a false positive error, then
  the prediction and the original tokenization are shown (the
  problematic token itself and some neighbouring tokens) and finally
  the complete sentence where the error occured (predicted and
  original version).


\item[X-fold cross-validation] the given data is split into X data
  set. X rounds of evaluation are run, in each round X-1 of these data
  sets are used for training and the remaining one is used for
  evaluation. Finally, the results of each round are averaged. (splits
  are made the same way as in 90-10 mode)

\begin{verbatim}
./runJTBDpackaged.sh x

-> usage: usage: JTBD x <sent-file> <tok-file> 
                        <cross-val-rounds> <predout-file> <errout-file>
\end{verbatim}

  \textit{<sentFile>}, \textit{<tokFile>}, \textit{<predout-file>} and
  \textit{<errorout-file>} are the same as in 90-10 split mode.
  \textit{<cross-val-round>} is the number of rounds for
  cross-validation. Typically, this might be set to 10.

\end{description}

\subsection{Training}

To train a tokenizer model, you need to provide some training material
(format: see above). The trained model can then be saved to disk and
used for tokenization.

\begin{verbatim}
./runJTBDpackaged.sh t

-> usage: JTBD t <sent-file> <tok-file> <model-file>
\end{verbatim}

\textit{<sent-file>} and \textit{<tok-file>} is the sentence splitted
and tokenized training data. \textit{<model-file>} is the file where
the resulting model is saved to.

\subsection{Prediction}

To employ the tokenizer, you need a trained model. Directory
\url{models} contains a pre-trained model for life-science documents.

\begin{verbatim}
./runJTBDpackaged.sh p

-> JTBD p JTBD p <inDir> <outDir> <model-file>
\end{verbatim}

\textit{<inDir>} is a directory of text documents which should be
tokenized. \textit{<model-file>} is the file where a previously
trained model was saved to. The processed texts with one sentence per
line and white spaces between the single tokens are written to
\textit{<outDir>}

\section{Background/Algorithms}
\label{sec:background}

JTBD is based on Conditional Random Fields (CRFs) \cite{lafferty01}, a
sequential learning algorithm. 

\section{Evaluation Studies and Available Models}

JTBD was developed and optimized for the bio-medical domain. However,
when training it on respective corpora, it may also be used for other
domains.


We have evaluated JTBD on our tokenization corpus which we compiled
for the bio-medical domain. It consists of data from (manually
annotated) material which we took from MedLine abstracts and a
modified version of
\textsc{PennBioIE}'s\footnote{\url{http://bioie.ldc.upenn.edu/}}
underlying tokenization. In the \textsc{PennBioIE} corpus, some purely
alphanumeric strings are divided into smaller tokens to support
\textsc{PennBioIE}'s entity annotation, especially in a common
annotation for variation events (e.g.  "S45F" with "S"=state\_original,
"45"=location, "F"=state\_altered).  Those splits were (manually)
undone to fit our tokenization guidelines.

Currently, our tokenization corpus comprises about 36000 sentences.
An accuracy of ACC=96.7\% is reached on this data using 10-fold
cross-validation.  You will find the model trained on this data in the
directory \url{models}.

If you run any evaluations on other data, we would be happy to learn about
your experiences with JTBD and evaluation results.

\section{Copyright and License}
% leave unchanged
This software is Copyright (C) 2008 Jena University Language \& Information
Engineering Lab (Friedrich-Schiller University Jena, Germany), and is
licensed under the terms of the Common Public License, Version 1.0 or (at
your option) any subsequent version.

The license is approved by the Open Source Initiative, and is
available from their website at \url{http://www.opensource.org}.

\bibliographystyle{alpha}
\bibliography{literature}


\end{document}
